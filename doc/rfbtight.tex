\documentclass[a4paper]{article}
\usepackage{geometry}
\geometry{verbose,a4paper,tmargin=20mm,bmargin=20mm,lmargin=30mm,rmargin=30mm}
\setlength{\parindent}{0pt}
\setlength{\parskip}{1ex plus 0.5ex minus 0.2ex}

\newcommand{\typestr}[1]{\textit{#1}}
\newcommand{\literal}[1]{\textit{\textbf{#1}}}

\begin{document}

\title{TightVNC Extensions to the RFB~Protocol
       \thanks{This document refers to RFB protocol versions 3.3, 3.7}}
\author{Constantin Kaplinsky\\
        TightVNC Group}
\date{Revision 1 --- July, 2006}
\maketitle

\tableofcontents

\newpage
\section{Overview}
This document describes various extensions to the RFB protocol used in
the TightVNC distribution and in other software compatible with
TightVNC. Only differences from the stardard RFB protocol are
discussed. The original RFB protocol specification (protocol versions
3.3, 3.7, 3.8) can be obtained here:
\begin{center}
\texttt{http://www.realvnc.com/docs/rfbproto.pdf}
\end{center}

At the moment of writing this document, TightVNC supports RFB protocol
version 3.3 (all TightVNC versions) and 3.7 (TightVNC 1.3.x). It does
not use the version 3.8 of the protocol, and it's not safe to assume
that the TightVNC-specific changes to the protocol 3.7 initialization
procedure can be used with the protocol 3.8. However, all encodings,
pseudo-encodings and protocol messages described in this document can
be safely used with the protocol 3.8.
% (pseudo-)encodings: 3.3+, messages: 3.7+ with sec-type == Tight

TightVNC introduces the following protocol changes:
\begin{itemize}
\item extended protocol initialization procedure,
\item new encodings,
\item new pseudo-encodings,
\item new protocol messages.
\end{itemize}

\newpage
\section{Protocol initialization}

\subsection{Protocol version 3.3}
\subsection{Protocol version 3.7}

\newpage
\section{Encodings}
\subsection{Tight}
\begin{verbatim}
Encoding type:     7
Name signature:    "TIGHT___"
Vendor signature:  "TGHT"
\end{verbatim}
Tight encoding provides efficient compression for pixel data.

The first byte of each Tight-encoded rectangle is a
\typestr{compression-control} byte:

\begin{tabular}{l|lc|l} \hline
No.\ of bytes & Type & [Value] & Description \\ \hline
1 & U8 &   & \typestr{compression-control} \\ \hline
\end{tabular}

\begin{quote}
The least significant four bits of \typestr{compression-control} byte
inform the client which zlib compression streams should be reset
before decoding the rectangle. Each bit is independent and corresponds
to a separate zlib stream that should be reset:

\begin{tabular}{l|l}
\hline
Bit & Description    \\
\hline
0   & if 1: reset stream 0 \\
1   & if 1: reset stream 1 \\
2   & if 1: reset stream 2 \\
3   & if 1: reset stream 4 \\
\hline
\end{tabular}

One of three possible compression methods are supported in Tight
encoding. These are \literal{BasicCompression},
\literal{FillCompression} and \literal{JpegCompression}. If the bit 7 (the
most significant bit) of \typestr{compression-control} byte is 0, then
the compression type is \literal{BasicCompression}. In that case, bits
7-4 (the most significant four bits) of \typestr{compression-control}
should be interpreted as follows:

\begin{tabular}{l|c|l}
\hline
Bits & Binary value & Description \\
\hline
5-4 & 00  & \literal{UseStream0} \\
    & 01  & \literal{UseStream1} \\
    & 10  & \literal{UseStream2} \\
    & 11  & \literal{UseStream3} \\
\hline
6   & 0   & --- \\
    & 1   & \literal{ReadFilterId} \\
\hline
7   & 0   & \literal{BasicCompression} \\
\hline
\end{tabular}

Otherwise, if the bit 7 of \typestr{compression-control} is set to 1,
then the compression method is either \literal{FillCompression} or
\literal{JpegCompression}, depending on other bits of the same byte:

\begin{tabular}{l|c|l}
\hline
Bits & Binary value & Description \\
\hline
7-4 & 1000 & \literal{FillCompression} \\
    & 1001 & \literal{JpegCompression} \\
    & any other & invalid \\
\hline
\end{tabular}

Note: \literal{JpegCompression} may only be used when
\typestr{bits-per-pixel} is either 16 or 32.

\end{quote}

The data following the \typestr{compression-control} byte depends on
the compression method.

\subsubsection*{\literal{FillCompression}}

-- If the compression type is "fill", then the only pixel value follows, in
 client pixel format (see NOTE 1). This value applies to all pixels of the
 rectangle.

\subsubsection*{\literal{JpegCompression}}

-- If the compression type is "jpeg", the following data stream looks like
 this:

   1..3 bytes:  data size (N) in compact representation;
   N bytes:     JPEG image.

 Data size is compactly represented in one, two or three bytes, according
 to the following scheme:

  0xxxxxxx                    (for values 0..127)
  1xxxxxxx 0yyyyyyy           (for values 128..16383)
  1xxxxxxx 1yyyyyyy zzzzzzzz  (for values 16384..4194303)

 Here each character denotes one bit, xxxxxxx are the least significant 7
 bits of the value (bits 0-6), yyyyyyy are bits 7-13, and zzzzzzzz are the
 most significant 8 bits (bits 14-21). For example, decimal value 10000
 should be represented as two bytes: binary 10010000 01001110, or
 hexadecimal 90 4E.

\subsubsection*{\literal{BasicCompression}}

-- If the compression type is "basic" and bit 6 of the compression control
 byte was set to 1, then the next (second) byte specifies "filter id" which
 tells the decoder what filter type was used by the encoder to pre-process
 pixel data before the compression. The "filter id" byte can be one of the
 following:

   0:  no filter ("copy" filter);
   1:  "palette" filter;
   2:  "gradient" filter.

-- If bit 6 of the compression control byte is set to 0 (no "filter id"
 byte), or if the filter id is 0, then raw pixel values in the client
 format (see NOTE 1) will be compressed. See below details on the
 compression.

-- The "gradient" filter pre-processes pixel data with a simple algorithm
 which converts each color component to a difference between a "predicted"
 intensity and the actual intensity. Such a technique does not affect
 uncompressed data size, but helps to compress photo-like images better. 
 Pseudo-code for converting intensities to differences is the following:

   P[i,j] := V[i-1,j] + V[i,j-1] - V[i-1,j-1];
   if (P[i,j] < 0) then P[i,j] := 0;
   if (P[i,j] > MAX) then P[i,j] := MAX;
   D[i,j] := V[i,j] - P[i,j];

 Here V[i,j] is the intensity of a color component for a pixel at
 coordinates (i,j). MAX is the maximum value of intensity for a color
 component.

-- The "palette" filter converts true-color pixel data to indexed colors
 and a palette which can consist of 2..256 colors. If the number of colors
 is 2, then each pixel is encoded in 1 bit, otherwise 8 bits is used to
 encode one pixel. 1-bit encoding is performed such way that the most
 significant bits correspond to the leftmost pixels, and each raw of pixels
 is aligned to the byte boundary. When "palette" filter is used, the
 palette is sent before the pixel data. The palette begins with an unsigned
 byte which value is the number of colors in the palette minus 1 (i.e. 1
 means 2 colors, 255 means 256 colors in the palette). Then follows the
 palette itself which consist of pixel values in client pixel format (see
 NOTE 1).

-- The pixel data is compressed using the zlib library. But if the data
 size after applying the filter but before the compression is less then 12,
 then the data is sent as is, uncompressed. Four separate zlib streams
 (0..3) can be used and the decoder should read the actual stream id from
 the compression control byte (see NOTE 2).

 If the compression is not used, then the pixel data is sent as is,
 otherwise the data stream looks like this:

   1..3 bytes:  data size (N) in compact representation;
   N bytes:     zlib-compressed data.

 Data size is compactly represented in one, two or three bytes, just like
 in the "jpeg" compression method (see above).

-- NOTE 1. If the color depth is 24, and all three color components are
 8-bit wide, then one pixel in Tight encoding is always represented by
 three bytes, where the first byte is red component, the second byte is
 green component, and the third byte is blue component of the pixel color
 value. This applies to colors in palettes as well.

-- NOTE 2. The decoder must reset compression streams' states before
 decoding the rectangle, if some of bits 0,1,2,3 in the compression control
 byte are set to 1. Note that the decoder must reset zlib streams even if
 the compression type is "fill" or "jpeg".

-- NOTE 3. The "gradient" filter and "jpeg" compression may be used only
 when bits-per-pixel value is either 16 or 32, not 8.

-- NOTE 4. The width of any Tight-encoded rectangle cannot exceed 2048
 pixels. If a rectangle is wider, it must be split into several rectangles
 and each one should be encoded separately.

\newpage
\section{Pseudo-encodings}
\subsection{NewFBSize}
\begin{verbatim}
Encoding type:     0xFFFFFF21
Name signature:    "NEWFBSIZ"
Vendor signature:  "TGHT"
\end{verbatim}

The NewFBSize pseudo-encoding is used to transmit information about
framebuffer size changes from server to client.

This pseudo-encoding was introduced in TightVNC, and was adopted for
use in RealVNC 4 (see the DesktopSize pseudo-encoding in the RFB 3.7+
specification). However, there is known incompatibility between
TightVNC and RealVNC in the way the server sends information about new
framebuffer size. TightVNC implementation requires that a NewFBSize
``rectangle'' must be the last one in a framebuffer update, while
RealVNC does not make such a requirement. As a result, TightVNC
viewers may incorrectly interpret the data stream received from
RealVNC 4 servers.

Hopefully, this incompatibility will be fixed in future versions of
TightVNC. Ideally, the server should send NewFBSize ``rectangles''
according to the TightVNC rules (with valid rectangle counter in the
FramebufferUpdate header), while the client should interpet it in
accordance to the DesktopSize pseudo-encoding description from
RealVNC's official RFB specification.

Below is the complete specification of the NewFBSize pseudo-encoding,
as it was introduced in TightVNC:

\subsubsection*{
\begin{center}
      Handling framebuffer size changes in the RFB protocol v.3\\
                 Technical Specification, revision 3
\end{center}
}

\begin{quote}
The following pseudo-encoding is used to transmit information about
framebuffer size changes from server to client:

\begin{center}
\texttt{\#define rfbEncodingNewFBSize 0xFFFFFF21}
\end{center}

A client supporting variable framebuffer size should send this
pseudo-encoding number in the encodings list within the SetEncodings
RFB message.

After a client informed the server about this capability, the server
may send a special rectangle within the FramebufferUpdate RFB message,
where the header contains encoding-type field set to the
rfbEncodingNewFBSize value. Fields width and height in this rectangle
header should be interpreted as new framebuffer size, fields
x-position and y-position are currently not used and should be set to
zero values by the server. Unlike true rectangle in framebuffer
updates, this one should not include any pixel data. This rectangle
must always be the last one in a FramebufferUpdate message, and a
client should treat it similarly to the LastRect marker, that is, it
should not expect more rectangles in this update, even if the
number-of-rectangles field in the FramebufferUpdate message header
exceeds current number of rectangles.

Immediately after receiving such a special rectangle, a client should
change its framebuffer size accordingly, and should interpret all
following framebuffer updates in context of the new framebuffer size.

Changing framebuffer size invalidates framebuffer contents, so the
server should mark all the framebuffer contents as changed.

FIXME: Allow zero-size framebuffer?
\end{quote}

\subsection{RichCursor}
\subsection{XCursor}
\subsection{PointerPos}
\subsection{CompressLevel}
\subsection{QualityLevel}

\subsection{LastRect}
\begin{verbatim}
Encoding type:     0xFFFFFF20
Name signature:    "LASTRECT"
Vendor signature:  "TGHT"
\end{verbatim}

\typestr{LastRect} enables \typestr{FramebufferUpdate} messages that
include less rectangles than was specified in the message header. For
example, VNC server can send a big conter like 0xFFFF as
\typestr{number-of-rectangles}, then arbitrary number of rectangles and
pseudo-rectangles (less than 0xFFFF). Finally, it sends
\typestr{LastRect} pseudo-rectangle which marks the end of current
\typestr{FramebufferUpdate} message.

\typestr{LastRect} pseudo-encoding is special -- it does not include
\typestr{x-position}, \typestr{y-position}, \typestr{width} and
\typestr{height} fields, and is not followed by any other data as
well. Only 4 bytes is sent in \typestr{LastRect}, while other
encodings and pseudo-encodings normally send 12 bytes as a rectangle
header.

To enable this pseudo-encoding, the client specifies
\typestr{LastRect} in the \typestr{SetEncodings} message. From that
moment, the server may use \typestr{LastRect} pseudo-encoding in some
of the framebuffer updates it will send.

% For each message, describe its place in the message sequence.

\newpage
\section{Protocol messages, file transfers (client to server)}

\subsection{FileListRequest}
\begin{verbatim}
Message type:      130
Name signature:    "FTC_LSRQ"
Vendor signature:  "TGHT"
\end{verbatim}

The client sends \typestr{FileListRequest} message to obtain the list
of files in a particular directory. The server is expected to respond
with exactly one \typestr{FileListData} message.

\begin{tabular}{l|lc|l} \hline
No.\ of bytes & Type & [Value] & Description \\ \hline
1 & U8  & 130 & \typestr{message-type} \\
1 & U8  &     & \typestr{flags} \\
2 & U16 &     & \typestr{dirname-length} \\
\typestr{dirname-length} & U8 array & & \typestr{dirname-string} \\
\hline\end{tabular}

The \typestr{flags} byte has the following format:

\begin{tabular}{l|l|l}
\hline
Bits & Binary value   & Description \\ \hline
3-0 & 0000            & do not compress file list data \\
    & 0001..1001      & use compression level 1..9 \\
    & 1010 or greater & invalid \\
\hline
4   & 0   & list of files and directories requested \\
    & 1   & list of directories only requested \\
\hline
\end{tabular}

The client can request compression only if enabled via the server's
\typestr{FileEnableCompression} message. Otherwise, bits 3-0 of
\typestr{flags} must be all zeroes (for no compression). See more
details in the specification for the \typestr{FileEnableCompression}
message.

% Add a note that specifying compression levels is not fatal here.

The \typestr{dirname-string} should be an absolute path represented in
Unix-style format: it should start with a forward slash
(``\verb|/|''), path components should be separated also with forward
slashes. There should be no trailing slash at the end of the path,
except for the root (highest level) directory which is denoted with a
single slash. Note that Windows-style path like
``\verb|c:\Program Files\TightVNC|'' should be sent as
``\verb|/c:/Program Files/TightVNC|''.

\newpage
\subsection{FileDownloadRequest}
\begin{verbatim}
Message type:      131
Name signature:    "FTC_DNRQ"
Vendor signature:  "TGHT"
\end{verbatim}

The client sends \typestr{FileDownloadRequest} message to start
downloading (receiving) a particular file from the server.

\begin{tabular}{l|lc|l} \hline
No.\ of bytes & Type & [Value] & Description \\ \hline
1 & U8  & 131 & \typestr{message-type} \\
1 & U8  &     & \typestr{flags} \\
2 & U16 &     & \typestr{filename-length} \\
4 & U32 &     & \typestr{initial-offset} \\
\typestr{filename-length} & U8 array & & \typestr{filename-string} \\
\hline\end{tabular}

The \typestr{flags} byte has the following format:

\begin{tabular}{l|l|l}
\hline
Bits & Binary value   & Description \\ \hline
3-0 & 0000            & do not compress file data \\
    & 0001..1001      & use compression level 1..9 \\
    & 1010 or greater & invalid \\
\hline
\end{tabular}

% \typestr{initial-offset}
% do not use compression when not enabled by the server

After the client sends \typestr{FileDownloadRequest}, the server is
expected to respond with one of the following:

\begin{itemize}

\item \typestr{FileLastRequestFailed} message, if there was an error
opening file on the server side.

\item Arbitrary number (including zero) of \typestr{FileDownloadData}
messages with file data, and then one more \typestr{FileDownloadData}
message with \typestr{raw-length} and \typestr{compressed-length}
fields set to zero. This is the normal scenario when there was no
error and download was not interrupted.

\item Arbitrary number (including zero) of \typestr{FileDownloadData}
messages with file data, and then one \typestr{FileDownloadFailed}
message. This happens when the server cannot complete file download
for some reason.

\item \dots

\end{itemize}

% Can FileDownloadFailed go first? -- probably yes.
% How zero-size files are transmitted?
% How to detect the beginning of the next file, when previous
%   download was cancelled with FileDownloadCancel?

The \typestr{filename-string} should be a file name with complete
absolute path represented in Unix-style format: it should start with a
forward slash (``\verb|/|''), path components should be separated also
with forward slashes. For example, Windows-style file name
``\verb|c:\Temp\Report.doc|'' should be sent as
``\verb|/c:/Temp/Report.doc|''.

\newpage
\subsection{FileUploadRequest}
\begin{verbatim}
Message type:      132
Name signature:    "FTC_UPRQ"
Vendor signature:  "TGHT"
\end{verbatim}

The client sends \typestr{FileUploadRequest} message to start
uploading (sending) a particular file to the server.

\begin{tabular}{l|lc|l} \hline
No.\ of bytes & Type & [Value] & Description \\ \hline
1 & U8  & 132 & \typestr{message-type} \\
1 & U8  &     & \typestr{flags} \\
2 & U16 &     & \typestr{filename-length} \\
4 & U32 &     & \typestr{initial-offset} \\
\typestr{filename-length} & U8 array & & \typestr{filename-string} \\
\hline\end{tabular}

The \typestr{flags} byte has the following format:

\begin{tabular}{l|l|l}
\hline
Bits & Binary value   & Description \\ \hline
3-0 & 0000            & file data is expected to be uncompressed \\
    & 0001..1001      & expected compression level (1..9) \\
    & 1010 or greater & invalid \\
\hline
\end{tabular}

% \typestr{initial-offset}
% do not use compression when not enabled by the server

Normally, the server does not respond to this message, so the client
can start sending \typestr{FileUploadData} messages immediately.
However, the server may send \typestr{FileUploadCancel} to make the
client cancel current upload. The client can cancel current upload by
sending \typestr{FileUploadFailed} message.

The \typestr{filename-string} should be a file name with complete
absolute path represented in Unix-style format: it should start with a
forward slash (``\verb|/|''), path components should be separated also
with forward slashes. For example, Windows-style file name
``\verb|c:\Temp\Report.doc|'' should be sent as
``\verb|/c:/Temp/Report.doc|''.

\newpage
\subsection{FileUploadData}
\begin{verbatim}
Message type:      133
Name signature:    "FTC_UPDT"
Vendor signature:  "TGHT"
\end{verbatim}

The client uses \typestr{FileUploadData} message to send each
successive portion of file data, as a part of file upload procedure.

\begin{tabular}{l|lc|l} \hline
No.\ of bytes & Type & [Value] & Description \\ \hline
1 & U8  & 133 & \typestr{message-type} \\
1 & U8  &     & \typestr{flags} \\
2 & U16 &     & \typestr{raw-length} \\
2 & U16 &     & \typestr{compressed-length} \\
\typestr{compressed-length} & U8 array & & \typestr{file-data} \\
\hline\end{tabular}

The \typestr{flags} byte has the following format:

\begin{tabular}{l|l|l}
\hline
Bits & Binary value   & Description \\ \hline
3-0 & 0000            & file data is not compressed \\
    & 0001..1001      & compression level used (1..9) \\
    & 1010 or greater & invalid \\
\hline
\end{tabular}

Last message per file (both \typestr{raw-length} and
\typestr{compressed-length} are set to zeroes):

\begin{tabular}{l|lc|l} \hline
No.\ of bytes & Type & [Value] & Description \\ \hline
1 & U8  & 133 & \typestr{message-type} \\
1 &     &     & \typestr{padding} \\
2 & U16 &   0 & \typestr{raw-length} \\
2 & U16 &   0 & \typestr{compressed-length} \\
4 & U32 &     & \typestr{modification-time} \\
\hline\end{tabular}

% do not use compression when not enabled by the server
% data is compressed with zlib
% format of the modification-time

\newpage
\subsection{FileDownloadCancel}
\begin{verbatim}
Message type:      134
Name signature:    "FTC_DNCN"
Vendor signature:  "TGHT"
\end{verbatim}

The client sends \typestr{FileDownloadCancel} message to make the
server break current download operation.

\begin{tabular}{l|lc|l} \hline
No.\ of bytes & Type & [Value] & Description \\ \hline
1 & U8  & 134 & \typestr{message-type} \\
1 &     &     & \typestr{padding} \\
2 & U16 &     & \typestr{reason-length} \\
\typestr{reason-length} & U8 array & & \typestr{reason-string} \\
\hline\end{tabular}

% reason-string may be empty

\newpage
\subsection{FileUploadFailed}
\begin{verbatim}
Message type:      135
Name signature:    "FTC_UPFL"
Vendor signature:  "TGHT"
\end{verbatim}

The client sends \typestr{FileUploadFailed} message to notify the
server that current upload operation will not be completed.

\begin{tabular}{l|lc|l} \hline
No.\ of bytes & Type & [Value] & Description \\ \hline
1 & U8  & 135 & \typestr{message-type} \\
1 &     &     & \typestr{padding} \\
2 & U16 &     & \typestr{reason-length} \\
\typestr{reason-length} & U8 array & & \typestr{reason-string} \\
\hline\end{tabular}

% reason-string may be empty

\newpage
\subsection{FileCreateDirRequest}
\begin{verbatim}
Message type:      136
Name signature:    "FTC_FCDR"
Vendor signature:  "TGHT"
\end{verbatim}

\typestr{FileCreateDirRequest} message requests the server to create a
new directory.

\begin{tabular}{l|lc|l} \hline
No.\ of bytes & Type & [Value] & Description \\ \hline
1 & U8  & 136 & \typestr{message-type} \\
1 &     &     & \typestr{padding} \\
2 & U16 &     & \typestr{dirname-length} \\
\typestr{dirname-length} & U8 array & & \typestr{dirname-string} \\
\hline\end{tabular}

The \typestr{dirname-string} should be an absolute path represented in
Unix-style format: it should start with a forward slash
(``\verb|/|''), path components should be separated also with forward
slashes. There should be no trailing slash at the end of the path.
Note that Windows-style path like
``\verb|c:\Temp\MyNewDir|'' should be sent as
``\verb|/c:/Temp/MyNewDir|''.

% All components of the path except the last one should exist in the
% server's file system?

\newpage
\subsection{FileDirSizeRequest}
\begin{verbatim}
Message type:      137
Name signature:    "FTC_DSRQ"
Vendor signature:  "TGHT"
\end{verbatim}

The client can send \typestr{FileDirSizeRequest} message to inquire
about the total size of all files within a particular directory on the
server (including files in sub-directories).

\begin{tabular}{l|lc|l} \hline
No.\ of bytes & Type & [Value] & Description \\ \hline
1 & U8  & 137 & \typestr{message-type} \\
1 &     &     & \typestr{padding} \\
2 & U16 &     & \typestr{dirname-length} \\
\typestr{dirname-length} & U8 array & & \typestr{dirname-string} \\
\hline\end{tabular}

The \typestr{dirname-string} should be an absolute path represented in
Unix-style format: it should start with a forward slash
(``\verb|/|''), path components should be separated also with forward
slashes. There should be no trailing slash at the end of the path,
except for the root (highest level) directory which is denoted with a
single slash. Note that Windows-style path like
``\verb|c:\Program Files\TightVNC|'' should be sent as
``\verb|/c:/Program Files/TightVNC|''.

\newpage
\subsection{FileRenameRequest}
\begin{verbatim}
Message type:      138
Name signature:    "FTC_RNRQ"
Vendor signature:  "TGHT"
\end{verbatim}

\typestr{FileRenameRequest} message requests the server to rename the
specified file or directory.

\begin{tabular}{l|lc|l} \hline
No.\ of bytes & Type & [Value] & Description \\ \hline
1 & U8  & 138 & \typestr{message-type} \\
1 &     &     & \typestr{padding} \\
2 & U16 &     & \typestr{old-name-length} \\
2 & U16 &     & \typestr{new-name-length} \\
\typestr{old-name-length} & U8 array & & \typestr{old-name-string} \\
\typestr{new-name-length} & U8 array & & \typestr{new-name-string} \\
\hline\end{tabular}

Both \typestr{old-name-string} and \typestr{new-name-string} should be
file names with complete absolute paths and should be represented in
Unix-style format: the strings should start with forward slashes
(``\verb|/|''), path components should be separated also with forward
slashes. For example, Windows-style file name
``\verb|c:\Temp\Report.doc|'' should be sent as
``\verb|/c:/Temp/Report.doc|''.

% All components of two paths except the last one should be the same?

\newpage
\subsection{FileDeleteRequest}
\begin{verbatim}
Message type:      139
Name signature:    "FTC_RMRQ"
Vendor signature:  "TGHT"
\end{verbatim}

\typestr{FileDeleteRequest} message requests the server to delete the
specified file or directory.

\begin{tabular}{l|lc|l} \hline
No.\ of bytes & Type & [Value] & Description \\ \hline
1 & U8  & 139 & \typestr{message-type} \\
1 &     &     & \typestr{padding} \\
2 & U16 &     & \typestr{filename-length} \\
\typestr{filename-length} & U8 array & & \typestr{filename-string} \\
\hline\end{tabular}

The \typestr{filename-string} should be a file name with complete
absolute path represented in Unix-style format: it should start with a
forward slash (``\verb|/|''), path components should be separated also
with forward slashes. For example, Windows-style file name
``\verb|c:\Temp\Report.doc|'' should be sent as
``\verb|/c:/Temp/Report.doc|''.

\newpage
\section{Protocol messages, file transfers (server to client)}

\subsection{FileListData}
\begin{verbatim}
Message type:      130
Name signature:    "FTS_LSDT"
Vendor signature:  "TGHT"
\end{verbatim}

The server sends \typestr{FileListData} message in response to
\typestr{FileListRequest} client message. \typestr{FileListData}
message lists files and sub-directories in a particular directory, and
includes information about file sizes and last modification times.

\begin{tabular}{l|lc|l} \hline
No.\ of bytes & Type & [Value] & Description \\ \hline
1 & U8  & 130 & \typestr{message-type} \\
1 & U8  &     & \typestr{flags} \\
2 & U16 &     & \typestr{number-of-files} \\
2 & U16 &     & \typestr{raw-length} \\
2 & U16 &     & \typestr{compressed-length} \\
8 $\times$ \typestr{number-of-files} & U32 array & & \typestr{file-properties-data} \\
\typestr{compressed-length} & U8 array & & \typestr{file-names-data} \\
\hline\end{tabular}

The \typestr{flags} byte has the following format:

\begin{tabular}{l|l|l}
\hline
Bits & Binary value   & Description \\ \hline
3-0 & 0000            & file list data is not compressed \\
    & 0001..1001      & compression level used (1..9) \\
    & 1010 or greater & invalid \\
\hline
4   & 0   & list of files and directories requested \\
    & 1   & list of directories only requested \\
\hline
5   & 0   & success \\
    & 1   & non-existent or inaccessible directory requested \\
\hline
\end{tabular}

In the \typestr{flags} byte:

\begin{itemize}
\item bit 4 duplicates the same bit from the corresponding
  \typestr{FileListRequest} message;
\item if bit 5 is set to 1, then \typestr{number-of-files},
  \typestr{raw-length} and \typestr{compressed-length} all should be
  set to zero values. In that case, \typestr{file-properties-data} and
  \typestr{file-names-data} should be empty;
\item compression bits (bits 3-0) should be set only if compression
  was requested by the client in the corresponding
  \typestr{FileListRequest} message.
\end{itemize}

\typestr{File-properties-data} array is sent as
\typestr{number-of-files} repetitions of the following structure:

\begin{tabular}{l|lc|l} \hline
No.\ of bytes & Type & [Value] & Description \\ \hline
4 & U32 &  & \typestr{file-size} \\
4 & U32 &  & \typestr{modification-time} \\
\hline\end{tabular}

In that structure, \typestr{file-size} is specified in bytes,
\typestr{modification-time} is an offset is seconds since the Epoch
(00:00:00 UTC, January 1, 1970).

\typestr{File-names-data} is a sequence of file and directory names
found in the requested directory. Names do not include paths.
Each individual record is terminated with a zero byte. The
\typestr{file-names-data} list should include
\typestr{number-of-files} individual records. If compression is
enabled (bits 3-0 of \typestr{flags} designate a valid compression
level), then \typestr{file-names-data} is compressed using zlib
library.

% the number-of-files may be 0; in that case ...
% how directories should be ditinguished
% what about symbolic links etc.
% describe raw-length and compressed-length

\newpage
\subsection{FileDownloadData}
\begin{verbatim}
Message type:      131
Name signature:    "FTS_DNDT"
Vendor signature:  "TGHT"
\end{verbatim}

The server uses \typestr{FileDownloadData} message to send each
successive portion of file data, as a part of file download procedure.

\begin{tabular}{l|lc|l} \hline
No.\ of bytes & Type & [Value] & Description \\ \hline
1 & U8  & 131 & \typestr{message-type} \\
1 & U8  &     & \typestr{flags} \\
2 & U16 &     & \typestr{raw-length} \\
2 & U16 &     & \typestr{compressed-length} \\
\typestr{compressed-length} & U8 array & & \typestr{file-data} \\
\hline\end{tabular}

Last message per file (both \typestr{raw-length} and
\typestr{compressed-length} are set to zeroes):

\begin{tabular}{l|lc|l} \hline
No.\ of bytes & Type & [Value] & Description \\ \hline
1 & U8  & 131 & \typestr{message-type} \\
1 &     &     & \typestr{padding} \\
2 & U16 &   0 & \typestr{raw-length} \\
2 & U16 &   0 & \typestr{compressed-length} \\
4 & U32 &     & \typestr{modification-time} \\
\hline\end{tabular}

The \typestr{flags} byte has the following format:

\begin{tabular}{l|l|l}
\hline
Bits & Binary value   & Description \\ \hline
3-0 & 0000            & file data is not compressed \\
    & 0001..1001      & compression level used (1..9) \\
    & 1010 or greater & invalid \\
\hline
\end{tabular}

Compression should be used only if it was requested by the client in
corresponding \typestr{FileListRequest} message.

% data is compressed with zlib
% format of the modification-time

\newpage
\subsection{FileUploadCancel}
\begin{verbatim}
Message type:      132
Name signature:    "FTS_UPCN"
Vendor signature:  "TGHT"
\end{verbatim}

The server sends \typestr{FileUploadCancel} message to make the client
break current upload operation.

\begin{tabular}{l|lc|l} \hline
No.\ of bytes & Type & [Value] & Description \\ \hline
1 & U8  & 132 & \typestr{message-type} \\
1 &     &     & \typestr{padding} \\
2 & U16 &     & \typestr{reason-length} \\
\typestr{reason-length} & U8 array & & \typestr{reason-string} \\
\hline\end{tabular}

% reason-string may be empty

\newpage
\subsection{FileDownloadFailed}
\begin{verbatim}
Message type:      133
Name signature:    "FTS_DNFL"
Vendor signature:  "TGHT"
\end{verbatim}

The server sends \typestr{FileDownloadFailed} message to notify the
client that current download operation will not be completed.

\begin{tabular}{l|lc|l} \hline
No.\ of bytes & Type & [Value] & Description \\ \hline
1 & U8  & 133 & \typestr{message-type} \\
1 &     &     & \typestr{padding} \\
2 & U16 &     & \typestr{reason-length} \\
\typestr{reason-length} & U8 array & & \typestr{reason-string} \\
\hline\end{tabular}

% reason-string may be empty

\newpage
\subsection{FileDirSizeData}
\begin{verbatim}
Message type:      134
Name signature:    "FTS_DSDT"
Vendor signature:  "TGHT"
\end{verbatim}

The server sends \typestr{FileDirSizeData} message in response to
\typestr{FileDirSizeRequest} client message. \typestr{FileDirSizeData}
message returns total size of all files within the requested directory
on the server (including files in sub-directories). The size in bytes
is returned as an unsigned 48-bit value.

\begin{tabular}{l|lc|l} \hline
No.\ of bytes & Type & [Value] & Description \\ \hline
1 & U8  & 134 & \typestr{message-type} \\
1 &     &     & \typestr{padding} \\
2 & U16 &     & \typestr{dir-size-high-bits} \\
4 & U32 &     & \typestr{dir-size-low-bits} \\
\hline\end{tabular}

\newpage
\subsection{FileLastRequestFailed}
\begin{verbatim}
Message type:      135
Name signature:    "FTS_RQFL"
Vendor signature:  "TGHT"
\end{verbatim}

The server sends \typestr{FileLastRequestFailed} message if it cannot
execute an operation requested by the client. It can be sent in
response to the following client messages:
\typestr{FileDownloadRequest}, \typestr{FileCreateDirRequest},
\typestr{FileRenameRequest}, \typestr{FileDeleteRequest}.

\begin{tabular}{l|lc|l} \hline
No.\ of bytes & Type & [Value] & Description \\ \hline
1 & U8  & 135 & \typestr{message-type} \\
1 & U8  &     & \typestr{message-type-of-request} \\
2 & U16 &     & \typestr{reason-length} \\
\typestr{reason-length} & U8 array & & \typestr{reason-string} \\
\hline\end{tabular}

% reason-string may be empty

\newpage
\subsection{FileEnableCompression}
\begin{verbatim}
Message type:      137
Name signature:    "FTS_CMPR"
Vendor signature:  "TGHT"
\end{verbatim}

This message controls compression in file transfer operations.
\typestr{Enable-flag} is non-zero (true) if the client should be
allowed to request and use compression for file transfers, or zero
(false) otherwise.

\begin{tabular}{l|lc|l} \hline
No.\ of bytes & Type & [Value] & Description \\ \hline
1 & U8  & 137 & \typestr{message-type} \\
1 & U8  &     & \typestr{enable-flag} \\
\hline\end{tabular}

\typestr{FileEnableCompression} message controls how clients send the
following messages:

\begin{itemize}
\item \typestr{FileListRequest}
\item \typestr{FileDownloadRequest}
\item \typestr{FileUploadRequest}
\item \typestr{FileUploadData}
\end{itemize}

If the client did not receive a \typestr{FileEnableCompression}
message from the server, or the most recent
\typestr{FileEnableCompression} message included \typestr{enable-flag}
set to false, then it must specify zero as compression level in the
messages listed above and, most importantly, may not compress file
data on uploading files.

Otherwise, if the most recent \typestr{FileEnableCompression} message
included non-zero \typestr{enable-flag}, then the client is allowed to
request non-zero compression level and may send compressed data in
\typestr{FileUploadData} messages.

This message should not have effect on file transfers that are already
in progress.

\begin{quote}
Note: this message was introduced to preserve compatibility with
servers that do not support compression in file transfers. Old servers
expect uncompressed data in \typestr{FileUploadData} messages but new
clients might send it compressed. Using
\typestr{FileEnableCompression} message makes sure the client will
never compress upload data unless enabled explicitly by the server.
\end{quote}

\newpage
\section{Protocol messages (client to server)}

\subsection{EnableContinuousUpdates}
\begin{verbatim}
Message type:      150
Name signature:    "CUC_ENCU"
Vendor signature:  "TGHT"
\end{verbatim}

This message either enables continuous updates for the specified
pixels, or disables continuous updates completely.

If \typestr{enable-flag} is set to true (non-zero), continuous updates
are enabled for the framebuffer area specified by
\typestr{x-position}, \typestr{y-position}, \typestr{width} and
\typestr{height}.

If \typestr{enable-flag} is set to false (zero), continuous updates
are to be completely disabled. In this case, the values of
\typestr{x-position}, \typestr{y-position}, \typestr{width} and
\typestr{height} are not used.

\begin{tabular}{l|lc|l} \hline
No.\ of bytes & Type & [Value] & Description \\ \hline
1 & U8  & 150 & \typestr{message-type} \\
1 & U8  &     & \typestr{enable-flag} \\
2 & U16 &     & \typestr{x-position} \\
2 & U16 &     & \typestr{y-position} \\
2 & U16 &     & \typestr{width} \\
2 & U16 &     & \typestr{height} \\
\hline\end{tabular}

Normally, the server would send \typestr{FramebufferUpdate} messages
only in response to corresponding \typestr{FramebufferUpdateRequest}
messages received from the client. In contrast, when continuous
updates are in effect, the server is allowed to send a
\typestr{FramebufferUpdate} message at any time, not waiting for
client requests.

Initially, contunuous updates are disabled since they violate the
original RFB protocol. To enable this mode, the client sends an
\typestr{EnableContinuousUpdates} message with non-zero
\typestr{enable-flag}. On receiving such a message, the server
immediately enables continuous updates for the area specified by
\typestr{x-position}, \typestr{y-position}, \typestr{width} and
\typestr{height}. The server will not send updates for pixels outside
of this area.

If continuous updates are already in effect, the client may always
change the coordinates of currently updated area by sending another
\typestr{EnableContinuousUpdates} message with non-zero
\typestr{enable-flag}. On receiving each next message, the server
discards old coordinates and re-enables continuous updates for the
newly specified area.

If \typestr{enable-flag} is set to zero, the server should disable
continuous updates completely and respond with an
\typestr{EndOfContinuousUpdates} message. After the client has
received an \typestr{EndOfContinuousUpdates}, it can be sure that
there will be no more updates without prior update requests.

The client may send a message with zero \typestr{enable-flag} even if
continuous updates were disabled already. In response, the server
should send an \typestr{EndOfContinuousUpdates} each time it is asked
to disable continuous updates. This behavior may be useful for
protocol synchronization purposes.

While continuous updates are in effect, the server completely ignores
incremental update requests (\typestr{FramebufferUpdateRequest} with
\typestr{incremental} flag set to non-zero). Such requests are ignored
regardless of the coordinates specified within the request. Thus, it's
not possible to request an incremental update for pixels outside of
current continuously updated area.

Non-incremental update requests (\typestr{FramebufferUpdateRequest}
with \typestr{incremental} flag set to zero) should work as usual.
That is, in response to such requests, the server should send the
entire contents of the specified area as soon as possible.

FIXME: Respect incremental update requests?

FIXME: Decide how to handle framebuffer size changes.

FIXME: Decide how to change pixel format while continuously updating.

See also:
\begin{itemize}
\item \typestr{EndOfContinuousUpdates} server-to-client message.
\end{itemize}

\newpage
\section{Protocol messages (server to client)}

\subsection{EndOfContinuousUpdates}
\begin{verbatim}
Message type:      150
Name signature:    "CUS_EOCU"
Vendor signature:  "TGHT"
\end{verbatim}

The server sends \typestr{EndOfContinuousUpdates} message in response
to each \typestr{EnableContinuousUpdates} message if and only if its
\typestr{enable-flag} was set to false (zero).

\begin{tabular}{l|lc|l} \hline
No.\ of bytes & Type & [Value] & Description \\ \hline
1 & U8  & 150 & \typestr{message-type} \\
\hline\end{tabular}

The message informs the client that it should not expect unsolicited
framebuffer updates any more. In other words, after this message,
there will be no framebuffer updates with no corresponding update
requests received from the client.

Note that this message should be sent each time the client is asking
to disable continuous updates, even if continuous updates were not
previously enabled.

FIXME: Allow unsolicited EndOfContinuousUpdates messages?

See also:
\begin{itemize}
\item \typestr{EnableContinuousUpdates} client-to-server message.
\end{itemize}

\end{document}
